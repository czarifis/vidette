
\vspace*{-0.4cm}
\section{Traditional Notebooks}



We will now describe how a data scientist would perform this analysis with a traditional interactive notebook that uses Python. The description will show the issues that arise.


\eat{
\noindent {\bf Setting up and interacting  with third-party utility libraries: Productivity and security concerns.}
}


\noindent {\bf Interacting  with third-party utility libraries: Productivity and security concerns.}


\eat{ Before even starting working on any of the afore-mentioned tasks the data analyst must make sure that all third-party python packages that will be used for retrieving and visualizing data, are installed. This step will either introduce a dependency between the analyst and some system administrator, or the analyst will need the appropriate access rights in order to perform the installation, typically, using a command line. The latter can be a security concern (especially, if the server on which the notebook operates, also hosts other applications) and can also lead to corrupted systems if not performed correctly. The complexity of this step increases as the number of utility libraries, our analyst wants to use, increases. \eat{More specifically, for cases when an analyst needs to access data stored in a MongoDB and a MySQL database, two drivers will need to be installed.}
}
\begin{figure*}[hbt!]
\centering
	\includegraphics[width=1\textwidth]{figures/reactive-processing2.pdf}
	\caption{Demonstration of reactive charts. The reader's selection automatically updates both charts.}
	\label{fig:reactive-data-processing}
	\vspace{-10pt}
\end{figure*}


In order to retrieve website access information from the database, the data analyst, needs to employ third-party Python drivers. Employing such drivers, involves reading lengthy documentation pages that describe the exposed API calls of the respective driver. After identifying the appropriate API calls, the analyst needs to use imperative logic that interacts with the library, in order to issue the appropriate query. \{During this step, the analyst has to also specify the credentials that will be used for accessing the database system. In most such libraries, the credentials have to be inlined into the method call that establishes the connection with that system. While this would not be a security concern in a python (or any other) application (since the code that includes the credentials would have been compiled into a binary file, thus hiding them), in a python notebook, the credentials would lie in plain sight for anyone, who has access to the notebook, to see. This security concern is also resolved in \projname.\} \costas{Is this point weak? Should I remove it?}

\noindent {\bf Data conversions.} 
After, establishing the connection with the database system, and issuing the queries, the analyst, is able to consume the results by using the internal, to the library, data types. After doing so, the analyst has to again read documentation pages in order to infer the data types required by the visualization library she selected, and manually perform the appropriate conversions in order to construct the first visualization, by invoking the appropriate renderer function. Note that data conversions have to take place every time the analyst wishes to integrate a third-party library in her analysis, so the amount of plumbing code required, can quickly skyrocket.


\noindent {\bf Limited interactive exploratory capabilities.}
The next step, is for the analyst to issue a query that counts the visitors per age group for a set of selected hours and construct a bar chart showing the result. The obtained dataset will also go through a machine learning package (for instance, the scikit-learn package \cite{scikit-learn}), that employs a linear regression model to predict the expected revenue during the selected hours. The predicted revenue will be plotted next to the actual revenue produced during these hours. Note that selecting a meaningful time frame, is of utmost importance, because it will affect all the remaining steps of this analysis, and ultimately the insight, the portal editor will gain, about potential ways to increase the revenue. 

Unfortunately, there is no straightforward way for selecting a meaningful time frame. The analyst will have to go through a process of trial and error, by issuing an arbitrary number of such queries and plotting the results, until she finds a set that produces valuable insight. Furthermore, even if the analyst goes through this process for the data corresponding to a particular day, the time frame selection she made, might not produce any valuable insight for the dataset corresponding to another day. Most importantly the non-technical reader cannot use the notebook to test similar hypotheses, she is limited to passively reading the ones that were hardcoded by the analyst. This aspect of introducing the human in the loop is improved dramatically with \projname\ notebooks.

\vspace*{-0.4cm}
\eat{
\subsection*{Reactive charts - Ideal user experience and functionality} Co-dependent reactive charts would be useful in this scenario. If the analyst was able to use a chart that allows the reader to select a particular range of hours by using the first plot, she could use that input in order to retrieve and plot only the user demographics for this particular range. Figure \ref{fig:reactive-data-processing} graphically depicts this process (note that the charts that appear side-by-side in Figure \ref{fig:reactive-data-processing} actually appear after the respective coding blocks that generated them in the notebook). The first figure shows the reader of the notebook, selecting a particular time frame; this causes the line chart to zoom in, thus showing only the selected time frame. After the reader performs the selection on the first chart, the subsequent bar chart is also updated thus only showing the age groups of the users that visited the portal during the selected hours. This feature adds useful exploratory capabilities to the notebook, which is of great value to readers. It enables them to further analyze the underlying dataset used by the notebook by simply interacting with the provided visualizations. 


\noindent {\bf Limitations of traditional interactive notebooks.} It is important to note that this feature cannot be implemented currently in interactive notebooks. Instead, the analyst would have to implement a full blown web application in order to enable non-technical users interact with visualizations in this way. This requires more time and effort as well as technical expertise with web frameworks that data analysts, might lack. This reactive behavior requires the implementation of actions that have to be invoked when particular mouse events take place on the pixels of the browser that correspond to the visualization. Additionally, such events have to take place in a specific order (mousedown-mousemove-mouseup) for the action to be invoked. The developer of such applications must install observers that listen for such events and then provide the application logic that asynchronously accesses the back-end database to retrieve new data based on the user's selection and then cause the appropriate mutations to the respective visualizations that depend on that data. 


\yannis{IMPORTANT: All figures have the problem that they do not look like interface pages, which sends the clear message to the reviewer that you have not even started yet. Make figures that show how the INTERFACE looks on the browser.}
\costas{Should I replace the figures with this: \url{https://czarifis.github.io/vidette-prototype/Thesis\%20Proposal\%20Sample\%20With\%20Tabls.html}}
}
 % Template and template instance figures
 \begin{figure*}[hbt!]
 \centering
 %
 %
 \begin{minipage}[c]{7cm}
 %
 \begin{minipage}[c]{7cm}
 \begin{code}
 \textbf{<\% let} readings = 
    SELECT count(time) as visits, time
    FROM (SELECT * FROM page_views pv 
   	     join visitors v 
          on pv.v_id = v.vid) AS joined_table
    GROUP BY time 
    ORDER BY time ASC \textbf{\%>};
 \end{code}
 \vspace*{-0.4cm}
 \subcaption{Data retrieval}
 \label{figure:first-running-example:data-retrieval}
 \vspace*{0.3cm}
 \end{minipage}
 
 
 
 \begin{minipage}[c]{7cm}
 \begin{code}
 readings = [
    \{visits: 15, time: '08:00'\}, 
    \{visits: 10, time: '09:00'\},
    \{visits: 25, time: '10:00'\},  ...]
 \end{code}
 \vspace*{-0.4cm}
 \subcaption{Query Result}
 \label{figure:running-example:query-result}
 \vspace*{0.3cm}
 \end{minipage}
 
 
 \begin{minipage}[c]{7cm}
 \begin{code}
 sources : [ \{ 
      driver   : "postgres", 
      host     : "edu.db.domain", 
      expose   : [ \{
       schema : 'website_info', 
       tables: [visits, page_views] \} ]
      port     : 5432, 
      username : "dbadmin" 
      password : 'myP@ss'
    \}] 
 \end{code}
 \vspace*{-0.4cm}
 \subcaption{DB Access Configuration file}
 \vspace*{0.3cm}
 \label{figure:source-config-file}
 \end{minipage}
 %
 \vspace*{0.6cm}
 \end{minipage}
 \hspace{2cm}
 \begin{minipage}[c]{6cm}
 \begin{minipage}[c]{7.5cm}
 \begin{code}
   \directive{unit}{highcharts} \{
     title: 'Visitor information' ,
     type: 'line',
     xAxis : \{ 
       labels : ['08:00','09:00'...],
       min : '08:00'
       max : '22:00'
     \}
     series: [\{ data: [ \{y:15\}, \{y:10\}...] \}]
   \} \directive{end unit}{}
 \end{code}
 \vspace*{-0.4cm}
 \subcaption{Unit with evaluated unit state}
 \vspace*{0.3cm}
 \label{figure:running-example:unit-body}
 \end{minipage}
 \begin{minipage}[c]{7.5cm}
 \begin{code}
   \directive{unit}{highcharts} \{
     title: 'Visitor information',
     type: 'line',
     xAxis : \{ 
       labels : [
         \directive{for}{reading \textbf{in} readings} 
           \directive{print}{reading.time} 
         \directive{end for}{}],
       min : \directive{bind}{min_time},
       max : \directive{bind}{max_time}
     \}
     series: [\{
       data: [ \directive{for}{reading \textbf{in} readings}
           \{
             y  : \directive{print}{reading.count}
           \}
         \directive{end for}{} ]
     \}]
   \} \directive{end unit}{}
 \end{code}
 \vspace*{-0.3cm}
 \subcaption{Template \texttt{temp\_view}}
 \vspace*{0cm}
 \label{figure:first-running-example:main-template}
 \end{minipage}
 \end{minipage}
 \vspace*{-0.3cm}
 \caption{Declarative Data retrieval, DB Configuration, evaluated unit state and Template describing unit state for running example}
 \vspace*{-0.3cm}
 \end{figure*}
