%------------------------------------------------------------------------------- 
% Math packages
%------------------------------------------------------------------------------- 

% The "amsmath" package provides advanced math extensions.
\usepackage{amsmath}

% The "amssymb" package adds new symbols to be used in math mode.
\usepackage{amssymb}

% The "amsthm" package adds the "proof" environment and "theoremstyle" command.
% \usepackage{amsthm}

% The "faktor" package adds the "faktor" macro for variable substitution (i.e. vulgar fractions). This depends on the "amssymb" package.
% \usepackage{faktor}

% The "semantic" package adds new macros for (PL-style) inference rules.
% \usepackage[inference]{semantic}

%------------------------------------------------------------------------------- 
% Figure packages
%------------------------------------------------------------------------------- 

% The "fancyvrb" package provides advanced customization of verbatim environments, such as font families, numbering lines, box borders etc.
% \usepackage{fancyvrb}

% The "graphicx" package allows including external graphic files.
\usepackage{graphicx}

% The "subfig" package allows multiple sub-figures within a single figure, where sub-figures can be separately captioned and labeled, e.g. Figure % 1.2(a). This is a replacement for the older "subfigure" package.
%\usepackage{subfig}

% Replaced the "subfig" package with the "subcaption" package
\usepackage{subcaption}

% HACK: The caption package (included by the subfig package) requires a counter for ACM's copyright box.
\newcounter{copyrightbox}

% The "float" package allows the "H" option for figures, which places a float % at a precise location.
\usepackage{float}

% The "caption" package allows captions for figures that are not actually in a floating environment (e.g. framed environment).
\usepackage{caption}

% The "mdframed" package creates framed regions that can break across pages.
\usepackage{mdframed}


% The "algorithm2e" package provides keywords for typesetting algorithms. The "noend" option disables the printing of the "end" keywords. Use "algomargin" to decrease the margins for all algorithms.

% Kevin: To resolve conflict of algorithm2e with other packages, a common problem with ACM template.
% See http://ergodicthoughts.blogspot.com/2009/06/latex-too-many-s-algorithm2e.html
% \makeatletter
% \newif\if@restonecol
% \makeatother
% \let\algorithm\relax
% \let\endalgorithm\relax
% \usepackage[noend,boxed]{algorithm2e}
% \setlength{\algomargin}{0.5em}

%------------------------------------------------------------------------------- 
% Layout packages
%------------------------------------------------------------------------------- 

% The "multirow" package allows table cells to span more than one row.
\usepackage{multirow}

% The "balance" package allows columns of the last page to be of equal height.
\usepackage{balance}

% The "fixltx2e" package prevents two-column figures from being placed out-of-order wrt regular (one-column) figures. 
% \usepackage{fixltx2e} 

% The "dblfloatfix" package allows two-column figures to be placed at the page's bottom. 
% \usepackage{dblfloatfix}

%------------------------------------------------------------------------------- 
% Whitespace packages
%------------------------------------------------------------------------------- 

% The "savetrees" package saves space on a page.
% \usepackage[all=normal,paragraphs=tight,floats=tight,bibnotes=tight]{savetrees}
% \usepackage[all=normal,paragraphs=tight,floats=tight,bibnotes=tight,bibliography=tight]{savetrees}

% The "setspace" package allows changing the inter-line spacing to be a multiple of the default line spacing.
% \usepackage{setspace}
% \setstretch{0.98}

% The "titlesec" package allows changing the whitespace around section headings.
% \usepackage[compact]{titlesec}

%------------------------------------------------------------------------------- 
% Font packages
%------------------------------------------------------------------------------- 

% The "beramono" package provides Bitstream Vera Mono, which has a bold typewritter fontface.
\usepackage[scaled]{beramono}
\usepackage[T1]{fontenc}

% The "courier" package provides Courier, which has a bold typewritter fontface.
%\usepackage{courier}

% The "upquote" package fixes tilde and quote in verbatim environments.
\usepackage{upquote}

%------------------------------------------------------------------------------- 
% Misc packages
%------------------------------------------------------------------------------- 

% The "optional" package allows multiple versions of the document via optional text.
\usepackage{optional}

% The "xstring" package allows switch/case conditionals.
\usepackage{xstring}

% The "xcolor" package allows colored text and backgrounds.
\usepackage[table]{xcolor}

% The "soul" package allows highlighting.
\usepackage{soul}

% The "ulem" package allows highlighting.
\usepackage[normalem]{ulem}

% The "tocloft" packages allows generating custom lists that are similar to table of contents, list of figures etc.
% \usepackage[subfigure]{tocloft}

% The "hyperref" package allows creating hyperlinks. Note that it must be the last package loaded, and will automatically includes the "url" package.
\usepackage{hyperref}

% The "hypcap" package fixes "hyperref" so that hyperlinks go to the top of a float (as opposed to its caption).
\usepackage[all]{hypcap}

%------------------------------------------------------------------------------- 
% Whitespace
%------------------------------------------------------------------------------- 

% Adjust whitespace above and below captions
% \addtolength{\abovecaptionskip}{-5pt}
% \addtolength{\belowcaptionskip}{-9pt}

% Adjust whitespace before/after floats
% \setlength{\textfloatsep}{10pt plus 1.0pt minus 2.0pt}
% \setlength{\floatsep}{6pt plus 1.0pt minus 1.0pt}

%------------------------------------------------------------------------------- 
% Macros
%------------------------------------------------------------------------------- 

% Define our own compact enumerate
\newenvironment{compact_enum}
{\setlength{\leftmargini}{1em}
\begin{enumerate}
  \setlength{\labelsep}{.3em} 
  \setlength{\itemsep}{.4em}
  \setlength{\parskip}{0pt}
  \setlength{\parsep}{0pt}}
{\end{enumerate}}

% Define our own compact itemize
\newenvironment{compact_item}
{\setlength{\leftmargini}{1em}
\begin{itemize}
  \setlength{\labelsep}{.3em} 
  \setlength{\itemsep}{.4em}
  \setlength{\parskip}{0pt}
  \setlength{\parsep}{0pt}}
{\end{itemize}}

% \newtheorem{theorem}{Theorem}[section]
% \newtheorem{lemma}[theorem]{Lemma}
% \newtheorem{proposition}[theorem]{Proposition}
% \newtheorem{corollary}[theorem]{Corollary}

% \newenvironment{proof}[1][Proof]{\begin{trivlist}
% \item[\hskip \labelsep {\bfseries #1}]}{\end{trivlist}}
% \newenvironment{definition}[1][Definition]{\begin{trivlist}
% \item[\hskip \labelsep {\bfseries #1}]}{\end{trivlist}}
% \newenvironment{example}[1][Example]{\begin{trivlist}
% \item[\hskip \labelsep {\bfseries #1}]}{\end{trivlist}}
% \newenvironment{remark}[1][Remark]{\begin{trivlist}
% \item[\hskip \labelsep {\bfseries #1}]}{\end{trivlist}}

% \newcommand{\qed}{\nobreak \ifvmode \relax \else
%       \ifdim\lastskip<1.5em \hskip-\lastskip
%       \hskip1.5em plus0em minus0.5em \fi \nobreak
%       \vrule height0.75em width0.5em depth0.25em\fi}


%------------------------------------------------------------------------------- 
% Database symbols
%------------------------------------------------------------------------------- 

\def\join{$\bowtie$}
\def\ojoin{\setbox0=\hbox{$\bowtie$}%
  \rule[-.02ex]{.25em}{.4pt}\llap{\rule[\ht0]{.25em}{.4pt}}}
\def\leftouterjoin{\mathbin{\ojoin\mkern-5.8mu\bowtie}}
\def\rightouterjoin{\mathbin{\bowtie\mkern-5.8mu\ojoin}}
\def\fullouterjoin{\mathbin{\ojoin\mkern-5.8mu\bowtie\mkern-5.8mu\ojoin}}
\def\semijoin{\mbox{$\mathrel{\raise1pt\hbox{\vrule height5pt depth0pt\hskip-1.5pt$>$\hskip -2.5pt$<$}}$}}
\def\antisemijoin{\overline{\semijoin}}

%------------------------------------------------------------------------------- 
% Grammar symbols (BNFs and Tree Grammars) 
%------------------------------------------------------------------------------- 

% Formatting commands for tree grammars
\newcommand{\gn}[1]  {\textit{#1}}              % (N)on-terminal
\newcommand{\gt}[1]  {\textbf{#1}}              % (T)erminal
\newcommand{\gl}[1]  {\texttt{\textbf{#1}}}     % (L)iteral
\newcommand{\gs}[1]  {\textit{#1}}              % (S)pecial construct
\newcommand{\gp}[0]  {$\rightarrow$}            % (P)roduction rule
\newcommand{\gd}[0]  { $|$ }                    % (D)isjunction
