
\section{Reactive Notebooks} 

\noindent {\bf Ideal reactive behavior and functionality.} Co-dependent reactive charts are tremendously useful in this scenario. If the analyst was able to introduce a chart that allows the reader to select a particular range of hours, by directly interacting with it, and use that input to retrieve and plot only the user demographics for this particular range, she could enable the reader to further explore the underlying data without requiring any code literacy. Figure \ref{fig:reactive-data-processing} graphically depicts this process (note that the charts that appear side-by-side in Figure \ref{fig:reactive-data-processing} actually appear after the respective coding blocks that generated them in the notebook). The first figure shows the reader of the notebook, selecting a particular time frame; this causes the line chart to zoom in, thus showing only the selected time frame. After the reader performs the selection on the first chart, the subsequent bar chart is also updated thus only showing the age groups of the users that visited the portal during the selected hours. This feature adds useful exploratory capabilities to the notebook, which is of great value to readers. It enables them to further analyze the underlying dataset used by the notebook by simply interacting with the provided visualizations. 


\noindent {\bf This kind of reactive behavior cannot be expressed in traditional interactive notebooks.} It is important to note that this feature cannot be implemented currently in interactive notebooks. Instead, the analyst would have to implement a full blown web application in order to enable non-technical users to interact with visualizations in this way. This requires more time and effort as well as technical expertise with web frameworks that data analysts, might lack. This reactive behavior requires the implementation of actions that have to be invoked when particular mouse events take place on the pixels of the browser that correspond to the visualization. Additionally, such events have to take place in a specific order (mousedown-mousemove-mouseup) for the action to be invoked. The developer of such applications must install observers that listen for such events and then provide the imperative application logic that asynchronously accesses back-end databases to retrieve new data based on the user's selection, process them appropriately and cause mutations to the respective visualizations that depend on them. 


\yannis{IMPORTANT: All figures have the problem that they do not look like interface pages, which sends the clear message to the reviewer that you have not even started yet. Make figures that show how the INTERFACE looks on the browser.}
\costas{Should I replace the figures with this: \url{https://czarifis.github.io/vidette-prototype/Thesis\%20Proposal\%20Sample\%20With\%20Tabls.html}}